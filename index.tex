\documentclass[a4paper, 11pt]{report}

\usepackage[utf8]{inputenc} % Définit l'encodage du document
\usepackage[T1]{fontenc} % Définit l'encodage de la fonte
\usepackage[francais]{babel} % Spécifie que le document est en français

\usepackage{lipsum}

\begin{document}
        \title{BE 3 - Extraction de connaissances}
        \author{Jordan \bsc{Abderrachid}\\ Thomas \bsc{Perrot}\\ Louis \bsc{Zawadski}}

        \maketitle


        \section{Exercice 1}
        \begin{table}[h!]
            \centering
        	\begin{tabular}{|c|c|c|c|c|c|}
        		\hline
        		Cluster & Outlook & Température & Humidity & Windy & Play \\
        		\hline
        		Cluster0 : & overcast & 72.0 & 90.0 & TRUE & yes \\
        		\hline
        		Cluster1 : & sunny & 85.0 & 85.0 & FALSE & no \\
        		\hline
        	\end{tabular}
        \end{table}
        \section{Exercice II}
        On applique l'algorithme \emph{SimpleKMeans} à la base de données \emph{weather.arff}. On obtient les résultats présentés dans le tableau ci-dessous.
        
        \begin{table}[h!]
        \centering
        \begin{tabular}{| l | l | l | l | l | l |}
        \hline
        Cluster & Outlook & Temperature & Humidity & Windy & Play \\
        \hline
        Cluster 0 & sunny & 75.9 & 84.1 & FALSE & yes \\
        \hline
        Cluster 1 & overcast & 69.4 & 77.2 & TRUE & yes \\
        \hline

        \end{tabular}
        \caption{Centroïdes calculés par l'algorithme \emph{SimpleKMeans} avec 2 clusters}
        \label{tab:exo_2}
        \end{table}
        
        \section{Exercice III}
        On peut remarquer que les centroïdes des clusters sont beaucoup plus proches dans le cas de \emph{SimpleKMeans}. Par ailleurs, l'algorithme Fartherst-First génère un cluster correspondant à Play=no et un autre à Play=yes. On peut donc en déduire que l'algorithme qui maximise le plus la dispersion inter-cluster est l'algorithme Farthest-First.
        
\end{document}
