\documentclass[a4paper, 11pt]{report}

\usepackage[utf8]{inputenc} % Définit l'encodage du document
\usepackage[T1]{fontenc} % Définit l'encodage de la fonte
\usepackage[francais]{babel} % Spécifie que le document est en français

\usepackage{lipsum}

\begin{document}
        \title{BE 3 - Extraction de connaissances}
        \author{Jordan \bsc{Abderrachid}\\ Thomas \bsc{Perrot}\\ Louis \bsc{Zawadski}}

        \maketitle

\section{Exercice I}

        On applique l'algorithme \emph{FarthestFirst} à la base de données \emph{weather.arff}. On obtient les résultats présentés dans le tableau ci-dessous.
	\begin{table}[h!]
	\centering
	\begin{tabular}{|c|c|c|c|c|c|}
		\hline
		Cluster & Outlook & Température & Humidity & Windy & Play \\
		\hline
		Cluster0 : & overcast & 72.0 & 90.0 & TRUE & yes \\
		\hline
		Cluster1 : & sunny & 85.0 & 85.0 & FALSE & no \\
		\hline
	\end{tabular}
	\caption{Centroïdes calculés par l'algorithme \emph{FarthestFirst} avec 2 clusters}
	\label{exo1}
\end{table}

        \section{Exercice II}
        On applique l'algorithme \emph{SimpleKMeans} à la base de données \emph{weather.arff}. On obtient les résultats présentés dans le tableau ci-dessous.
        
        \begin{table}[h!]
        \centering
        \begin{tabular}{| l | l | l | l | l | l |}
        \hline
        Cluster & Outlook & Temperature & Humidity & Windy & Play \\
        \hline
        Cluster 0 & sunny & 75.9 & 84.1 & FALSE & yes \\
        \hline
        Cluster 1 & overcast & 69.4 & 77.2 & TRUE & yes \\
        \hline

        \end{tabular}
        \caption{Centroïdes calculés par l'algorithme \emph{SimpleKMeans} avec 2 clusters}
        \label{tab:exo_2}
        \end{table}
        
        \section{Exercice III}
        On peut remarquer que les centroïdes des clusters sont beaucoup plus proches dans le cas de \emph{SimpleKMeans}. Par ailleurs, l'algorithme Fartherst-First génère un cluster correspondant à Play=no et un autre à Play=yes. On peut donc en déduire que l'algorithme qui maximise le plus la dispersion inter-cluster est l'algorithme Farthest-First.
        
	\section{Exercice V}
	On applique les successivement les algorithmes \emph{FarthestFirst } et \emph{SimpleKMeans} à la base de données \emph{weather.arff} en chochant l'option "Store clusters for visualization". On obtient les résultats présentés dans le tableau ci-dessous.

	\begin{table}[h!]
		\centering
		\begin{tabular}{|l|c|l|c|}
			\hline
			Outlook : & overcast  & Outlook : & rainy \\
			Windy : & TRUE & Windy : & TRUE \\
			\hline
			Outlook : & overcast & Outlook : & rainy \\
			Windy : & FALSE & Windy : & FALSE \\
			\hline
		\end{tabular}
		\caption{Combinaisons de valeurs de outlook et windy pour lesquelles toutes les instances appartiennent au cluster 0 pour la méthode \emph{FarthestFirst}.}
		\label{tab:exo5-1}
	\end{table}


	\begin{table}[h!]
		\centering
		\begin{tabular}{|l|c|l|c|}
			\hline
			Outlook : & sunny & Outlook : & X \\
			Windy : & FALSE & Windy : & X \\
			\hline
		\end{tabular}
		\caption{Combinaisons de valeurs de outlook et windy pour lesquelles toutes les instances appartiennent au cluster 1 pour la méthode \emph{FarthestFirst}.}
		\label{tab:exo5-1}
	\end{table}

        \section{Exercice VI}
        On change le paramètre seed de l'algorithme \emph{SimpleKMeans}. On remarque que les clusters sont initialisés avec play=yes pour l'un et play=no pour l'autre.
        \begin{table}[h!]
        \centering
        \begin{tabular}{| l | l | l | l | l | l |}
        \hline
        Cluster & Outlook & Temperature & Humidity & Windy & Play \\
        \hline
        Cluster 0 & sunny & 75 & 84.1 & FALSE & yes \\
        \hline
        Cluster 1 & sunny & 71 & 77.2 & TRUE & no \\
        \hline

        \end{tabular}
        \caption{Centroïdes calculés par l'algorithme \emph{SimpleKMeans} avec un seed différent}
        \label{tab:exo_6}
        \end{table}
        
        \section{Exercice VII}
        On visualise les résultats obtenus par la simulation précédente. On remarque que les combinaisons de valeurs suivantes appartiennent au cluster 0 :
        \begin{table}[h!]
        \centering
        \begin{tabular}{| c | c |}
         \hline
         Outlook & Windy \\
         \hline
         sunny & false \\
         overcast & false \\
         rainy & false \\
         \hline
        
        \end{tabular}
        \caption{Couples Outlook-Windy appartenant au cluster 0}
        \label{tab:exo7_1}
        \end{table}
        
        Les couples suivants appartiennent au cluster 1 : 
        \begin{table}[h!]
        \centering
        \begin{tabular}{| c | c |}
         \hline
         Outlook & Windy \\
         \hline
         sunny & true \\
         rainy & true \\
         \hline
        
        \end{tabular}
        \caption{Couples Outlook-Windy appartenant au cluster 1}
        \label{tab:exo7_2}
        \end{table}
        
        \section{Exercice VIII}
        On compare les résultats obtenus avec \emph{Farthest-First} et \emph{SimpleKMeans}. On peut remarquer que les couples de valeurs outlook-windy suivantes ont des clusters différents selon la classification : 
        \begin{table}[h!]
        \centering
        \begin{tabular}{| c | c |}
         \hline
         Outlook & Windy \\
         \hline
         sunny & false \\
         overcast & true \\
         rainy & true\\
         \hline
        
        \end{tabular}
        \caption{Couples Outlook-Windy ayant des clusters différents}
        \label{tab:exo8}
        \end{table}
        
        \section{Exercice IX}
        On affiche côte-à-côte les graphiques représentant l'appartenance au cluster et l'attribut play pour les couples de valeurs Outlook-Windy.
        On remarque des différences pour les couples de valeur suivants :
        \begin{table}[h!]
        \centering
        \begin{tabular}{| c | c |}
         \hline
         Outlook & Windy \\
         \hline
         sunny & false \\
         sunny & true \\
         \hline
        
        \end{tabular}
        \caption{Couples Outlook-Windy ayant des valeurs de play et des clusters différents}
        \label{tab:exo9}
        \end{table}
        
\end{document}
