\documentclass[a4paper, 11pt]{report}

\usepackage[utf8]{inputenc} % Définit l'encodage du document
\usepackage[T1]{fontenc} % Définit l'encodage de la fonte
\usepackage[francais]{babel} % Spécifie que le document est en français

\begin{document}
        \title{BE 3 - Extraction de connaissances}
        \author{Jordan \bsc{Abderrachid}\\ Thomas \bsc{Perrot}\\ Louis \bsc{Zawadski}}

        \maketitle

        \section{Exercice II}
        On applique l'algorithme \emph{SimpleKMeans} à la base de données \emph{weather.arff}. On obtient les résultats présentés dans le tableau ci-dessous.
        
        \begin{table}[h!]
        \centering
        \begin{tabular}{| l | l | l | l | l | l |}
        \hline
        Cluster & Outlook & Temperature & Humidity & Windy & Play \\
        \hline
        Cluster 0 & sunny & 75.9 & 84.1 & FALSE & yes \\
        \hline
        Cluster 1 & overcast & 69.4 & 77.2 & TRUE & yes \\
        \hline

        \end{tabular}
        \caption{Centroïdes calculés par l'algorithme \emph{SimpleKMeans} avec 2 clusters}
        \label{tab:exo_2}
        \end{table}
        
\end{document}
